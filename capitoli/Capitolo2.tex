\newpage
\section{Reti Neurali Artificiali}

\subsection{Definizione}
Le reti neurali artificiali sono sistemi di elaborazione inspirati dal funzionamento dei sistemi biologici caratterizzati dalla capacità di apprendere compiti complessi con lo scopo di catture i principi di base dalla complessità biologica, piuttosto che riprodurla per intero.

\subsection{Architettura}
Una rete neurale artificiale ha un'architettura ad elaborazione distribuita e parallela che utilizza:
\begin{itemize}
    \item Semplici unità di elaborazione
    \item Elevato grado di interconnessione
    \item Semplici messaggi numerici scalari (segnale analogico)
    \item Interazioni adattive tra elementi
\end{itemize}

Questa struttura deve possedere delle importanti proprietà, ossia:
\begin{itemize}
    \item Apprendere dall'esperienza ma con la possibilità di rispondere anche in situazioni nuove
    \item Assenza di conoscenze esplicite riguardanti il problema
    \item Adattabilità in situazioni caotiche e/o caotiche (presenza di disturbi nei dati)
\end{itemize}

\subsection{Elementi di Base}
Gli elementi fondamentali di una rete neurale artificiale sono:
\begin{itemize}
    \item Neuroni (Unità di Elaborazione): fungono da rilevatori che segnalano attraverso la loro attivazione
    \item Reti Neurali: connettono, coordinano, amplificano e selezionano i pattern di attivazione sui neuroni
    \item Apprendimento: organizza le reti per eseguire compiti e per sviluppare modelli interni dell'ambiente.
\end{itemize}

\subsubsection{Il Neurone}
Un neurone formale è un modello matematico che cerca di catturare gli aspetti fondamentali del funzionamento neuronale.\\
Questi elementi comunicano con gli altri neuroni unidirezionalmente o bidirezionalmente con gli altri neuroni tramite delle connessioni, le quali hanno un peso indicante la forza della connessione.\\
La connessione può essere rappresentata da un valore positivo (eccitatore) o negativo (inibitorio).\\
L'input che un neurone riceve da ciascuno dei neuroni a sé collegati è rappresento moltiplicando il segnale proveniente da quel neurone per il preso sulla connessione.\\
L'input totale del neurone è la sommatoria delle attivazione che il neurone riceve da tutti gli altri neuroni.\\
Lo stato di attivazione finale viene calcolato attraverso la funzione di attivazione (o di output) del neurone. Normalmente si utilizza la funzione sigmoide, la quale presenta sempre valori compresi nell'intervallo [0,1].\\
Matematicamente, possiamo definire l'input totale al neurone \textbf{i} come:
\[net_i=\sum_j^N(w_{ij} x_j(-b))\]
dove:
\begin{itemize}
    \item \(net_i\) è l'input totale al neurone \(i\), ossia la sommatoria di tutti gli input
    \item \(x_j\) è il segnale in arrivo dal neurone \(j\)
    \item \(w_{ij}\) è il peso sinaptico della connessione tra il neurone \(j\) e il neurone ricevente \(i\)
    \item \(b\) è la soglia o \textit{bias} del neurone
\end{itemize}

\subsubsection{Reti di Neuroni}
L'architettura della rete serve per identificare l'organizzazione in gruppi o in strati dei neuroni, ossia la topologia della rete, e il modo in cui i neuroni sono collegati tra loro, ossia lo schema di connettività.
\href{https://www.youtube.com/watch?v=aircAruvnKk}{Cliccando qui}, si potrà visionare un video del canale Youtube \href{https://www.youtube.com/channel/UCYO_jab_esuFRV4b17AJtAw}{\textit{"3Blue1Brown"}} che spiega abbastanza nel dettaglio cosa sia una rete neurale.

\paragraph{Topologia della Rete}
\begin{itemize}
    \item Lo strato di input è formato dalle unità che ricevono direttamente informazioni dall'ambiente
    \item Lo strato di output è formato dalle unità di output finale della rete
    \item Le unità che non si trovano in contatto diretto con input o output sono chiamate unità nascoste. Le unità di input e output sono invece chiamate unità visibili
    \item Quando ci sono più di due strati nascosti, parleremo di rete profonda (\textit{Deep Network})
\end{itemize}

\paragraph{Schema di Connettività}
\begin{itemize}
    \item Reti \textit{feed-forward}: ci sono solo connessioni unidirezionali da unità di input a unità nascoste a unità di output (\textit{bottom-up})
    \item Reti ricorrenti: ci sono connessioni bidirezionali in cui l'attivazione può propagarsi all'indietro (\textit{top-down} o \textit{feedback})
    \item Reti interamente ricorrenti: come le reti ricorrenti, ma ci sono anche connessioni intra-stato (tra neuroni dello stesso livello)
\end{itemize}

\subsection{Differenze tra Reti Neurali Biologiche e Artificiali}
Nel tipico neurone artificiale, risultano mancanti: l'organizzazione spaziali dei contatti sinaptici, differenziazione tra neuroni eccitatori e inibitori, tipi diversi di sinapsi e la mancanza della dinamica del neurone con potenziali di azione.\\
Quest'ultima mancanza può essere brevemente approfondita cliccando \href{https://it.wikipedia.org/wiki/Rete_neurale_spiking}{qui}, link del Professore riportato nelle slide.\\\\
Nella tipica rete artificiale, risultano mancanti: la struttura laminare e l'organizzazione colonnare, l'organizzazione in mappe topografiche.
Rimane inoltre la differenza di scala tra la rete biologica e quella artificiale, infatti, normalmente, nel cervello ci sono \(10^{11}\) neuroni, di cui ogni singolo neurone comunica direttamente con circa \(10^{4}\) altri neuroni.
