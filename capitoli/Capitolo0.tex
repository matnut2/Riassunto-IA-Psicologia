\newpage
\begin{abstract}
    Questo documento rappresenta una parte del materiale di studio fornito dal Prof. Zorzi e dal Prof. Testolin nell'Anno Accademico 2021-2022 e non ha lo scopo di sostituire le spiegazioni dei docenti, ma di fornire un breve riassunto finalizzato al ripasso.\\ In questo documento verranno inoltre riportati alcuni dei link consigliati dal Professore e che hanno lo scopo di approfondire alcuni argomenti del corso; oltre ai link appena citati, ne ho aggiunti altri che ritengo possano essere utili per una migliore comprensione di alcuni argomenti che sono inerenti esclusivamente alla psicologia e non all'informatica.
    Nel caso fossero presenti dei refusi o errori vari, potete scrivermi all'indirizzo mail \href{mailto:matteo.solda.1@studenti.unipd.it}{matteo.solda.1@studenti.unipd.it} o su Telegram a @Matt2000eo.\\
    Buono studio!
\end{abstract}

\newpage
\tableofcontents

\newpage
\section{Libro di Testo}
Il libro utilizzato per il corso è disponibile in formato E-Book \href{https://infoscience.epfl.ch/record/63947}{qui}.

\section{Introduzione}
Nel 1950, Alan Turing si propose come problema di capire se ciò che è meccanico può manifestare un comportamento intelligente.\\
La nascita dell'IA, sia come termine che come disciplina, viene fatta risalire al un convegno fondativo del 1956 tenutosi al Dartmouth College (USA).\\
Per definizione, l'Intelligenza Artificiale è definita come "lo studio di agenti intelligenti che percepiscono il loro ambiente e producono azioni volte a massimizzare la probabilità di successo nel raggiungere i loro scopi". \\
Ma cosa sono nel concreto le IA? \href{https://www.raiplay.it/video/2018/08/Intelligenze-artificiali-29082018-f27a3b85-338c-46ba-97ae-2574c47a40a4.html}{Clicca qui} per vedere un servizio di Superquark intitolato "\textit{Intelligenze Artificiali}" registrato in UniPD.
\subsection{Due Prospettive}
Lo scopo dell'IA può essere definito come quello di costruire degli "agenti intelligenti", e quindi si studia come si potrebbero riprodurre in un computer i processi mentali. Questo può portare a due prospettive:
\begin{itemize}
    \item Prospettiva Ingegneristica: costruire dispositivi più intelligenti dotati di funzioni e capacità che si avvicinino all'intelligenza umana, con lo scopo di \textbf{imitare} l'intelligenza umana. Risulta quindi di minore importanza capire cosa succeda realmente nel cervello.
    \item Prospettiva delle Scienze Cognitive: costruire e testare \textbf{ipotesi specifiche} sui meccanismi che si presentano in modo da avvicinarsi il più possibile al comportamento umano, anche quando questo non è ottimale.
\end{itemize}

\subsection{IA Debole e IA Forte}
L'intelligenza artificiale stretta (\textit{Narrow Artificial Intelligence}), nota anche come IA debole, si riferisce a qualsiasi intelligenza artificiale in grado di eguagliare o superare un essere umano in un compito strettamente definito e strutturato.\\\\
L'intelligenza artificiale generale (\textit{Artificial General Intelligence - AGI}), nota anche come IA forte, dovrebbe consentire alle macchine di applicare conoscenze e abilità in diversi contesti anche nuovi. Tutt'oggi, questo obiettivo non è stato realizzato e non è detto che sarà mai possibile.

\subsection{IA Simbolica e IA Neurale}
In base alla definizione di "intelligenza", possiamo distinguere due tipi di Intelligenza Artificiale:
\begin{itemize}
    \item IA Simbolica (Classica): se l'intelligenza viene vista come la manipolazione di strutture simboliche di rappresentazione della conoscenza, l'IA  si può realizzare in un computer scrivendo un programma.
    \item IA Neurale: se l'intelligenza viene vista come l'attività di neuroni nel formare complesse reti neurali, allora l'IA si può realizzare simulando delle reti neurali.
\end{itemize}
