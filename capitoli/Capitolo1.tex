\newpage
\section{La Computazione Neurale}

\subsection{Definizione}
La Computazione Neurale (\textit{Neural Computing}) è l'elaborazione dell'informazione eseguite da reti di neuroni, biologici o artificiali. Essa rappresenta un modello alternativo alla computazione digitale.\\
Le reti neurali biologiche sono formate da neuroni reciprocamente influenzati attraverso delle connessioni (sinapsi) che li collegano.\\
Ogni neurone rileva un certo insieme di condizioni e segnala ciò che ha rilevato attraverso la sua \textit{frequenza di scarica}. Essi possono ricevere segnali da altri neuroni e formare quindi degli strati di rilevatori più complessi.\\
Le interazioni tra neuroni sono adattive e si modificano attraverso l'apprendimento.

\subsection{Codifica dell'Informazione nel Neurone Biologico}
Normalmente, il neurone si sintonizza su di uno stimolo preferito, che quando è presente produce la più forte tra le varie risposte.\\
Il \textit{campo recettivo} del neurone è quella porzione di spazio sensoriale che attiva la risposta del neurone.
